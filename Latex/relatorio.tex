\documentclass[a4paper, 12pt]{article}
\usepackage[brazilian]{babel} %pacote babel para tweaks em relaçao à linguagem
\usepackage[utf8]{inputenc} %acentos e carácteres específicos
\usepackage[scaled]{helvet}
\renewcommand\familydefault{\sfdefault}
\usepackage[T1]{fontenc}
\usepackage{csquotes} %csquotes auxilia babel
\usepackage[
bottom=2cm,
top=3cm,
left=3cm,
right=2cm,
]{geometry}  %geometry para definição das margens
\usepackage{indentfirst} %primeiro parágrafo "indentado"
\begin{document}
\title{\textbf{Teoria dos Grafos}\\ \small{Grafos: Tipos, matrizes e graus de um grafo.}}
\author{Leon Ferreira Bellini\\
	\small{\textbf{RA\@ 22218002--8}}\\
	e\\
   Guilherme Ormond Sampaio\\
   \small{\textbf{RA\@ 22218007--7}}
}
\date{}
\maketitle
\section{Introdução}
Um grafo representa de forma simples e eficaz as interdependências entre os elementos de um conjunto. A utilidade da aplicação de grafos tem se mostrado presente em variadas micro e macroestruturas, como a podendo representar cidades, linhas ferroviárias, fluxo de dados ou até mesmo circuitos eletrônicos. A forma de como é representado pode ser dita como didática e facilmente compreendida. Foi elaborado um algorítmo para que fossem aplicados os diversos conceitos aprendidos em aula, este algorítmo sendo desenvolvido em \textbf{Python}, linguagem a qual permite que o programa funcione em variadas plataformas como \textit{GNU/Linux} e \textit{Windows}. 
\section{O conceito de grafos}
\subsection{Vértices, arestas e função de incidência {$\varphi$}}
Para que haja um grafo é necessário obter o conjunto de \textbf{vértices}, dado por $\textbf{N} = \{v1, v2, \ldots, vn\}$, estes também chamados de nós, os quais representam os componentes, itens ou elementos que receberão, ou não, ligações entre si, tais ligações chamadas de \textbf{arestas}, as quais também formam um conjunto, dado por $\textbf{E} = \{e1, e2, \ldots, en\}$.

Existe, também, uma \textbf{função de incidência} $ \varphi $ (phi), a qual associa um par de vértices para cada aresta, como por exemplo, $ \varphi(e) = \{ a, b \}$, ou seja, \textit{e} incide em \textit{a} e \textit{b}.

Pode se dizer, então, que um grafo pode ser representado como: \[ G = (N, M, \varphi) \]

\subsubsection{Casos Específicos}

\begin{itemize}
	\item\textbf{Grafo Finito}:  \textbf{V} e \textbf{E} são finitos.
	\item\textbf{Grafo Trivial}: Possui apenas um vértice.
	\item\textbf{Grafo Nulo}: Não possui arestas.
\end{itemize}

\subsubsection{Laços}
\subsection{Representação matricial de um grafo}
\subsubsection{Matriz de adjacência}
\subsubsection{Matriz de incidência}
\subsection{Grafo simples}
\subsection{Grafo bipartido}
\subsubsection{Bipartido completo}
\subsection{Graus de um grafo}
\subsection{Sequência gráfica}
\section{Conclusão}

\end{document}
