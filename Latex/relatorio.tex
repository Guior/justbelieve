\documentclass[a4paper, 12pt]{article}
\usepackage[brazilian]{babel} %pacote babel para tweaks em relaçao à linguagem
\usepackage{amsmath}
\usepackage[utf8]{inputenc} %acentos e carácteres específicos
\usepackage[scaled]{helvet}
\renewcommand\familydefault{\sfdefault}
\usepackage[T1]{fontenc}
\usepackage{csquotes} %csquotes auxilia babel
\usepackage[
bottom=2cm,
top=3cm,
left=3cm,
right=2cm,
]{geometry}  %geometry para definição das margens
\usepackage{indentfirst} %primeiro parágrafo "indentado"
\usepackage{tkz-graph} %tkz-graph para criação de grafos
\usepackage{multicol}%para várias colunas

\begin{document}
\title{\textbf{Teoria dos Grafos}\\ \small{Grafos: Tipos, matrizes e graus de um grafo.}}
\author{Leon Ferreira Bellini\\
	\small{\textbf{RA\@ 22218002--8}}\\
	e\\
   Guilherme Ormond Sampaio\\
   \small{\textbf{RA\@ 22218007--7}}
}
\date{}
\maketitle
\section{Introdução}
Um grafo representa de forma simples e eficaz as interdependências entre os elementos de um conjunto. A utilidade da aplicação de grafos tem se mostrado presente em variadas micro e macroestruturas, como a podendo representar cidades, linhas ferroviárias, fluxo de dados ou até mesmo circuitos eletrônicos. A forma de como é representado pode ser dita como didática e facilmente compreendida. Foi elaborado um algorítmo para que fossem aplicados os diversos conceitos aprendidos em aula, este algorítmo sendo desenvolvido em \textbf{Python}, linguagem a qual permite que o programa funcione em variadas plataformas como \textit{GNU/Linux} e \textit{Windows}. 
\section{O conceito de grafos}
\subsection{Vértices, arestas e função de incidência {$\varphi$}}
Para que haja um grafo é necessário obter o conjunto de \textbf{vértices}, dado por $\textbf{N} = \{v1, v2, \ldots, vn\}$, estes também chamados de nós, os quais representam os componentes, itens ou elementos que receberão, ou não, ligações entre si, tais ligações chamadas de \textbf{arestas}, as quais também formam um conjunto, dado por $\textbf{E} = \{e1, e2, \ldots, en\}$.

Existe, também, uma \textbf{função de incidência} $ \varphi $ (phi), a qual associa um par de vértices para cada aresta, como por exemplo, $ \varphi(e) = \{ a, b \}$, ou seja, \textit{e} incide em \textit{a} e \textit{b}.

Pode se dizer, então, que um grafo pode ser representado como: \[ G = (N, M, \varphi) \]

\subsubsection{Casos Específicos}

\begin{itemize}
	\item\textbf{Grafo Finito}:  \textbf{V} e \textbf{E} são finitos.
	\item\textbf{Grafo Trivial}: Possui apenas um vértice.
	\item\textbf{Grafo Nulo}: Não possui arestas.
\end{itemize}

\subsubsection{Laços}
\subsection{Representação matricial de um grafo}
\subsection{Matriz de adjacência |A| e matriz incidência |M| }

    Para fins de representação não-gráfica dos grafos utiliza-se da matriz de adjacência ou da matriz de incidência.
    
\subsubsection{Matriz de adjacência |A|}

    A matriz de adjacência consiste em uma matriz com \textbf{n}-linhas e \textbf{n}-colunas, sendo \textbf{n} cada vértice do grafo. Os elementos da matriz representam as arestas do grafo, portanto, trata-se da relação de conexões entre seus vértices.
    
    \vspace{0.5in}
    
    \begin{multicols}{2}
        \begin{tikzpicture}[scale=1]
            \Vertex[x=1, y=4]{V1}
            \Vertex[x=4, y=4.5]{V2}
            \Vertex[x=2, y=2.5]{V3}
            \Vertex[x=0, y=2.2]{V4}
            \Vertex[x=3.5, y=0]{V5}
            \Vertex[x=4.2, y=2.3]{V6}
            \Edges(V2,V6,V5,V4,V4,V3,V1)
            \Loop[dist=3cm,dir=WE, style={thick,-}](V4)
            \tikzset{EdgeStyle/.append style = {bend right}}
            \Edge(V2)(V1)
            \tikzset{EdgeStyle/.append style = {bend right}}
            \Edge(V1)(V2)
        \end{tikzpicture}
        
        \begin{tabular}{ccccccc}
                & V1 & V2 & V3 & V4 & V5 & V6 \\
                V1 & 0  & 2  & 1  & 0  & 0  & 0  \\
                V2 & 2  & 0  & 0  & 0  & 0  & 1  \\
                V3 & 1  & 0  & 0  & 1  & 0  & 0  \\
                V4 & 0  & 0  & 1  & 1  & 1  & 0  \\
                V5 & 0  & 0  & 0  & 1  & 0  & 1  \\
                V6 & 0  & 1  & 0  & 0  & 1  & 0 
        \end{tabular}
        
    \end{multicols}
    
\subsubsection{Matriz de incidência |M|}

    A matriz de adjacência \textbf{M(G)} é uma matriz com \textbf{|V|} linhas e \textbf{|E|} colunas, tal que seus elementos (\textbf{aij}) representam quantas vezes a aresta \textbf{ej} incide no vértice \textbf{Vi}.
    
    \vspace{0.5in}
    
    Exemplo:
    
    \begin{multicols}{2}
        \begin{center}
        G:
        \end{center}
        \begin{tikzpicture}[scale=1]
            \tikzset{LabelStyle/.style= {fill = white}}
            \Vertex[x=1, y=4]{V1}
            \Vertex[x=4, y=4.5]{V2}
            \Vertex[x=2, y=2.5]{V3}
            \Vertex[x=0, y=2.2]{V4}
            \Vertex[x=3.5, y=0]{V5}
            \Vertex[x=4.2, y=2.3]{V6}
            \Edge[label=e3](V1)(V3)
            \Edge[label=e4](V3)(V4)
            \Edge[label=e6](V4)(V5)
            \Edge[label=e7](V5)(V6)
            \Edge[label=e8](V6)(V2)
            \Loop[dist=3cm,dir=WE, style={thick,-}, label=e5](V4)
            \tikzset{EdgeStyle/.append style = {bend right}}
            \Edge[label=e1](V2)(V1)
            \tikzset{EdgeStyle/.append style = {bend right}}
            \Edge[label=e2](V1)(V2)
        \end{tikzpicture}
        
        \begin{center}
        M(G):
        \end{center}
        \begin{tabular}{ccccccccc}
            & e1 & e2 & e3 & e4 & e5 & e6 & e7 & e8 \\
            V1 & 1  & 1  & 1  & 0  & 0  & 0  & 0  & 0  \\
            V2 & 1  & 1  & 0  & 0  & 0  & 0  & 0  & 1  \\
            V3 & 0  & 0  & 1  & 1  & 0  & 0  & 0  & 0  \\
            V4 & 0  & 0  & 0  & 1  & 2  & 1  & 0  & 0  \\
            V5 & 0  & 0  & 0  & 0  & 0  & 1  & 1  & 0  \\
            V6 & 0  & 0  & 0  & 0  & 0  & 0  & 1  & 1 
        \end{tabular}
    \end{multicols}
    
    \vspace{0.5in}
    
    Note que, por uma aresta estar conectada sempre em dois pontos, a soma dos elementos de cada coluna é 2.
    
\subsection{Grafo simples}

Um grafo é simples se não possui laços ou arestas múltiplas, logo, no exemplo de um grafo tendo $\textbf{V} = \{v1, v2, v3, v4\}$ e $\textbf{E} = \{e1, e2, e3, e4, e5\}$, sua matriz de adjacência \textbf{A} pode ser dada por:
 

    \begin{multicols}{2}
	    \[
        \begin{tikzpicture}[scale=1]
            \Vertex[x=1, y=4]{V1}
            \Vertex[x=4, y=4]{V2}
            \Vertex[x=4, y=1]{V3}
            \Vertex[x=1, y=1]{V4}
	    \Edges(V1,V2,V2,V3,V3,V4,V4,V1,V4,V2,V1,V3)
        \end{tikzpicture}
\]
\[
	A(G) = 
\begin{bmatrix}
	0	&1	&1 	&1 \\
	1	&0	&1	&1 \\
	1	&1	&0	&1  \\
	1	&1	&1	&0
\end{bmatrix}
\]
\end{multicols}
\textbf{obs}: Note que a diagonal principal da matriz indica se existem ou não, laços. Um número maior que 1 indicaria que existem mais de uma aresta ligando os dois vértices. Podem existir vértices de grau 0 em grafos simples.
\subsubsection{Grafo bipartido}
\subsubsection{Bipartido completo}
\subsection{Graus de um grafo}
\subsection{Sequência gráfica}
\section{Conclusão}

\end{document}
