\documentclass[a4paper, 12pt]{article}
\usepackage[brazilian]{babel} %pacote babel para tweaks em relaçao à linguagem
\usepackage[utf8]{inputenc} %acentos e carácteres específicos
\usepackage[scaled]{helvet}
\renewcommand\familydefault{\sfdefault}
\usepackage[T1]{fontenc}
\usepackage{csquotes} %csquotes auxilia babel
\usepackage[
bottom=2cm,
top=3cm,
left=3cm,
right=2cm,
]{geometry}  %geometry para definição das margens
\usepackage{indentfirst} %primeiro parágrafo "indentado"
\begin{document}
\title{\textbf{Teoria dos Grafos}\\ \small{Grafos: Tipos, matrizes e graus de um grafo.}}
\author{Leon Ferreira Bellini\\
	\small{\textbf{RA\@ 22218002--8}}\\
	e\\
   Guilherme Ormond Sampaio\\
   \small{\textbf{RA\@ 22218007--7}}
}
\date{}
\maketitle
\section{Introdução}
\section{O conceito de grafos}
\subsection{Vértices, arestas e função de incidência {$\varphi$}}
\subsubsection{Laços}
\subsection{Grafo simples}
\subsection{Grafo bipartido}
\subsubsection{Bipartido completo}
\subsection{Graus de um grafo}
\subsection{Matriz de adjacência |A| e matriz incidência |M| }
\subsection{Sequência gráfica}
\section{Conclusão}
\end{document}
